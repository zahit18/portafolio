
\documentclass{article}

\title{Fundamentos del Desarrollo Móvil y Arquitectura Móvil}
\author{Gutierrez Hernandez Kovin Zahit \\ Grupo: 3B}
\date{}
\begin{document}
	
	\maketitle
	
	\section*{Información de la Tarea}
	\begin{itemize}
		\item \textbf{Profesor:} Ray Brunett Parra Galaviz
		\item \textbf{Tarea:} Primer avance portafolio
	\end{itemize}
	
	\section*{Fundamentos del Desarrollo móvil/Arquitectura móvil}
	
	La arquitectura móvil se refiere al diseño estructural y la organización de componentes de software dentro de una aplicación móvil. En el contexto del desarrollo de aplicaciones móviles, abarca los diversos patrones, técnicas y metodologías empleadas para diseñar, construir y mantener aplicaciones móviles sólidas, eficientes y escalables. La adopción de una arquitectura móvil bien definida desempeña un papel importante a la hora de facilitar el rápido desarrollo de aplicaciones móviles de alta calidad que ofrecen experiencias de usuario excepcionales, una integración perfecta con diversos servicios y un rendimiento excelente en múltiples plataformas y dispositivos.
	
	Una arquitectura móvil sólida aborda varios desafíos, incluida la gestión de datos, la escalabilidad, la seguridad, el diseño de la interfaz de usuario (UI), la solidez, la compatibilidad de la plataforma y la adaptabilidad a las distintas capacidades de los dispositivos. Las arquitecturas móviles de alta calidad tienen en cuenta la amplia diversidad de dispositivos móviles y sistemas operativos, lo que permite a los desarrolladores ofrecer experiencias óptimas ajustando el diseño, la navegación y el rendimiento en función de las capacidades del dispositivo. Al atender a estos factores, una arquitectura móvil tiene como objetivo ofrecer experiencias fluidas y que satisfagan las demandas únicas del ecosistema móvil.
	
	Uno de los enfoques principales de la arquitectura móvil es diseñar e implementar una integración perfecta con sistemas backend y servicios de terceros, como API REST, servicios en la nube y bases de datos. Para lograr esto, la arquitectura móvil debe implementar varios mecanismos de sincronización, almacenamiento en caché y acceso a datos que permitan modos de operación tanto en línea como fuera de línea. Una arquitectura móvil adecuada también incorporará autenticación, autorización y almacenamiento seguro de datos confidenciales para garantizar que la información del usuario esté protegida contra el acceso no autorizado y posibles amenazas a la seguridad.
	
	\section*{Aplicaciones nativas y aplicaciones no nativas}
	
	\textbf{Apps nativas:} son las que se desarrollan específicamente para un sistema operativo, y se tienen que instalar en el dispositivo.
	
	\textbf{Apps no nativas:} consisten en un solo desarrollo que se puede desplegar en varios sistemas operativos y también son fácilmente accesibles a través de un navegador. Hasta hace poco, las apps nativas aún tenían cierta ventaja a la hora del funcionamiento y la calidad, y en algunos casos su aplicación era necesaria. Pero una tendencia de mejora en las aplicaciones no nativas las ha convertido en la opción preferible para realizar todo tipo de soluciones para el cliente final. Son más rápidas y económicas de desarrollar, más fáciles de sacar al mercado, y más escalables.
	
	Algunas apps no nativas conocidas son Instagram, Tesla, Walmart y Facebook, aplicaciones de cinco estrellas con las que se puede hacer prácticamente lo mismo que con una aplicación nativa para Android o iOS.
	
	\section*{Patrones de diseño para aplicaciones móviles}
	
	En el desarrollo de aplicaciones móviles, existen varios patrones y marcos de arquitectura móvil comúnmente utilizados por los desarrolladores, que incluyen:
	
	\begin{itemize}
		\item \textbf{Modelo-Vista-Controlador (MVC):} un patrón de diseño ampliamente utilizado que separa la lógica de la aplicación en tres componentes interconectados: Modelo (datos), Vista (presentación) y Controlador (procesamiento de entradas). Esta arquitectura puede simplificar el desarrollo y el mantenimiento al garantizar que cada componente esté organizado y enfocado en sus responsabilidades particulares.
		
		\item \textbf{Model-View-ViewModel (MVVM):} un patrón arquitectónico popular para aplicaciones móviles que separa la interfaz de usuario y las capas de lógica empresarial en componentes separados, llamados Modelo (datos), Vista (presentación) y ViewModel (lógica de presentación). Este patrón tiene como objetivo reducir la complejidad del código, promover la reutilización del código y mejorar la capacidad de prueba.
		
		\item \textbf{Model-View-Intent (MVI):} un patrón de arquitectura que introduce una capa de intención adicional para manejar las interacciones del usuario y otros eventos de la aplicación, mientras que las capas Modelo y Vista permanecen enfocadas en el almacenamiento y la presentación de datos, respectivamente. Este patrón enfatiza el flujo de datos unidireccional y la comunicación unidireccional, lo que puede mejorar la previsibilidad y el mantenimiento de la aplicación.
		
		\item \textbf{Patrones de comportamiento:} estos patrones describen cómo los objetos se comunican entre sí y cómo colaboran para lograr un objetivo común. Promueven un acoplamiento flexible y mejoran la flexibilidad de su sistema orientado a objetos.
	\end{itemize}
	
\end{document}
